\documentclass[a4paper,12pt, spanish]{report}
\usepackage[utf8]{inputenc}
\usepackage{enumitem} %permite el uso de letras para enumerar
\usepackage{graphicx} %para las imágenes
\usepackage{float} %para fijar las imágenes
\usepackage{booktabs} %para fijar las imágenes
\usepackage{wrapfig} %para que el texto se acomode entre las imagenes
\usepackage[spanish]{babel}
\renewcommand{\arraystretch}{2} % Escala la altura de las filas
\usepackage{multirow}

\usepackage{amsmath}%para entornos de alineación
\usepackage{amsfonts}%para las letras lindas de matemática
\usepackage{xfrac}%para fracciones chiquitas
\allowdisplaybreaks
\setlength{\jot}{8pt}%modifica el interlineado
\usepackage[font=footnotesize,labelfont=bf]{caption}

\usepackage{pgfplots}

\usepackage{tikz} %Librería para gráficos
\usetikzlibrary{calc, arrows.meta, positioning}

\usepackage[a4paper, %margenes de pagina
  left=2.5cm,
  right=2.5cm,
  top=2cm,
  bottom=2cm,
  includehead
]{geometry}

\usepackage{fancyhdr}
\pagestyle{fancy}
\lhead{UTN-FRC}
\chead{Dispisitivos Electronicos I}
\rhead{3R2}
\cfoot{\thepage}
\setlength{\headwidth}{\textwidth} % Hace que el ancho del encabezado coincida con el ancho del texto
\setlength{\headheight}{15pt}  % Ajusta la altura del encabezado
\setlength{\headsep}{20pt}     % Ajusta la separación entre el encabezado y el contenido

\usepackage{titlesec}
\titleformat{\chapter}[display]{\normalfont\Large\bfseries}{}{0pt}{}
\titlespacing*{\chapter}{10pt}{-45pt}{10pt}

\usepackage{etoolbox} 
\makeatletter
\patchcmd{\chapter}{\thispagestyle{plain}}{\thispagestyle{fancy}}{}{} %Muestra encabezado en las paginas con \chapter
\makeatother

%Comandos de fake section y fake sub section, para poder agregar secciones al indice
\newcommand{\fs}[1]{%
  \par\refstepcounter{section}% Increase section counter
  \sectionmark{#1}% Add section mark (header)
  \addcontentsline{toc}{section}{\protect\numberline{\thechapter.\alph{section}}#1}% Add section to ToC
}
\newcommand{\fss}[1]{%
  \par\refstepcounter{subsection}% Increase subsection counter
  \subsectionmark{#1}% Add subsection mark (header)
  \addcontentsline{toc}{subsection}{\protect\numberline{\alph{subsection}}#1}% Add subsection to ToC
}

\usepackage{afterpage}
\newcommand\myemptypage{
  \newpage
  \null
  \thispagestyle{empty}
  \addtocounter{page}{-1}
  \newpage
}

\usepackage{circuitikz}

\title{%
\setlength{\headwidth}{\textwidth} % Hace que el encabezado tenga el mismo ancho que el contenido
\setlength{\headheight}{15pt}  % Ajusta la altura del encabezado
\setlength{\headsep}{10pt}     % Ajusta la separación entre el encabezado y el contenido
  \fontsize{25}{0}\selectfont Universidad Tecnológica Nacional \\
  \fontsize{22}{30}\selectfont Dispositivos Electrónicos I \\
  \fontsize{20}{25}\selectfont Trabajo Practico N° 1
}
\author{
  Santino Noccetti - 405927\\
  Franco Palombo - 401910\\
}
\date{08 / 04 / 2025}

\begin{document}

  \maketitle

  \myemptypage

  \tableofcontents
  \thispagestyle{plain}

  \myemptypage

  \chapter{Introducción}
    En este trabajo de laboratorio se propuso el análisis y medición de un circuito puramente resistivo, constando así de 3
    partes fundamentales.

    El primero es el calculo tanto de la corriente como el voltaje que es atravesada/caída por cada resistor,
    teniendo así que utilizar las leyes de Kirchoff, las leyes de Ohm, y la posterior deducción de un sistema de
    ecuaciones para el calculo de todos los parámetros.

    Luego, utilizando el programa LTSpice, se arma el mismo circuito, se simula y mide nuevamente la corriente y el
    voltaje, teniendo así una forma de verificar los valores ideales que calculamos previamente utilizando las leyes
    de Kirchoff y Ohm.

    Como una actividad anexa, el profesor propuso que utilizando un generador de funciones y un osciloscopio, podamos
    experimentar con corriente alterna, y ver como se comportan dichos componentes resistivos cuando se les aplican ese
    tipo de corrientes. También, la actividad anexa sirve el propósito de aprender a usar herramientas nuevas.

    Por ultimo, en la facultad se realiza el procedimiento de armado del circuito y toma de mediciones del mismo.
    La facultad nos proporciono diferentes elementos, entre ellos, una fuente de alimentación de corriente continua,
    registrada como F-44, un generador de funciones GwInstek SFG-2120, registrado como G-174, y por ultimo, el
    osciloscopio LEADER 8041 de 40MHz, registrado como O-98 en conjunto con una sonda. Para la medición de parámetros
    continuos, alternos y corrientes, el grupo proporciono un voltímetro UNI-T modelo UT33A+.

  \myemptypage
  \chapter{Actividad Practica}
    \section{Cálculos}
    \vspace{-0.6cm}
      \begin{figure}[h]
        \centering
        \begin{minipage}{0.7\textwidth}
          \centering
          \begin{circuitikz}[american voltages]
              % nodos
              \draw
                (0, -1) to [V=$V_s$, on grid, invert]                   (0, 3)
                        to [short, -, i>^=$I_t$, on grid]               (1, 3)
                        to [R=$R_1$, v=$V_{R_1}$, on grid]              (4, 3)
                        to [short, -*, on grid]                         (5, 3)
                        to [short, -, on grid]                          (8, 3)
                        to [short, -, i>^=$I_3$, on grid]               (8, 2)
                        to [R=$R_3$, v=$V_{R_3}$, on grid]              (8, 0)
                        to [short, -, on grid]                          (8, -1)
                        to [short, -*, on grid]                         (5, -1)
                        to [short, -, on grid]                          (0, -1)
                        to [short, -, on grid]                          (0, -1)
                (5, 3)  to [short, -, i>^=$I_2$, on grid]               (5, 2)
                        to [R=$R_2$, v=$V_{R_2}$, on grid]              (5, 0)
                        to [short, -, on grid]                          (5, -1)
                ;
          \end{circuitikz}
        \end{minipage}
        \centering
        \begin{minipage}{0.2\textwidth}
          \centering
          \begin{align*}
            V_s &= 10V\\
            R_1 &= 10K\Omega\\
            R_2 &= 4K7\Omega\\
            R_3 &= 3K3\Omega
          \end{align*}
        \end{minipage}
      \end{figure}

      Ecuaciones del Circuito:
      \begin{gather*}
        \begin{cases}
          M_1: &-V_s + V_{R_1} + V_{R_2} = 0\\
          M_2: &-V_s + V_{R_1} + V_{R_3} = 0\\
          &I_2 + I_3 = I_t
        \end{cases}
      \end{gather*}

      Sabiendo que, por ley de Ohm, la caída de tensión en cualquier resistor es:
      \begin{equation}
        V_R = R I
      \end{equation}
      donde $R$ es la resistencia de dicho resistor, e $I$ la intensidad de corriente que pasa por ese resistor,
      podemos entonces despejar la caída de tensión en la $R_1$ reemplazando la caída de tensión de la $R_2$ en la ec.
      de la malla $M_1$:
      \begin{figure}[!h]
        \centering
        \begin{minipage}{0.6\textwidth}
          \centering
          \begin{align*}
            -V_s + V_{R_1} + V_{R_2} &= 0\\
            -V_s + V_{R_1} + R_2 I_2 &= 0\\
            -V_s + V_{R_1} + R_2 (I_t - I_3) &= 0\tag*{\hfill(\ref{i2_sub})}\\
            -V_s + V_{R_1} + R_2 \left(\frac{V_{R_1}}{R_1} - \frac{V_{R_3}}{R_3}\right) &= 0\tag*{\hfill(\ref{it_sub})(\ref{i3_sub})}\\
            -V_s + V_{R_1} + V_{R_1} \frac{R_2}{R_1} - (V_s - V_{R_1}) \frac{R_2}{R_3} &= 0\tag*{\hfill(\ref{vr3_sub})}\\
            V_{R_1} + V_{R_1} \frac{R_2}{R_1} + V_{R-1} \frac{R_2}{R_3} &= V_s \left(1 + \frac{R_2}{R_3}\right)\\
            V_{R_1} \left(1 + \frac{R_2}{R_1} + \frac{R_2}{R_3}\right) &= V_s \left(1 + \frac{R_2}{R_3}\right)\\
            V_{R_1} &= \frac{V_s \left(1 + \frac{R_2}{R_3}\right)}{1 + \frac{R_2}{R_1} + \frac{R_2}{R_3}}
          \end{align*}
        \end{minipage}
        \vline
        \centering
        \begin{minipage}{0.35\textwidth}
          \centering
          \textit{Cálculos Auxiliares}
          \begin{align}
            I_2 + I_3 &= I_t\nonumber\\
            I_2 &= I_t - I_3\tag{a}
            \label{i2_sub}
          \end{align}
          \begin{align}
            V_{R_1} &= R_1 I_t\nonumber\\
            \frac{V_{R_1}}{R_1} &= I_t\tag{b}
            \label{it_sub}
          \end{align}
          \begin{align}
            V_{R_3} &= R_3 I_3\nonumber\\
            \frac{V_{R_3}}{R_3} &= I_3\tag{c}
            \label{i3_sub}
          \end{align}
          \begin{align}
            -V_s + V_{R_1} + V_{R_3} &= 0\nonumber\\
            V_{R_3} &= V_s - V_{R_1}\tag{d}
            \label{vr3_sub}
          \end{align}
        \end{minipage}
      \end{figure}

      Con la ecuación para $V_{R_1}$ podemos despejar $V_{R_2}$, y por consecuente $V_{R_3}$, que son iguales debido a
      que la caída de tensión para resistores en paralelo es la misma. Entonces, reemplazando valores obtenemos:
      \begin{figure}[!h]
        \centering
        \begin{minipage}{0.4\textwidth}
          \begin{align*}
            V_{R_1} &= \frac{V_s \left(1 + \frac{R_2}{R_3}\right)}{1 + \frac{R_2}{R_1} + \frac{R_2}{R_3}}\\
            V_{R_1} &= \frac{10V \left(1 + \frac{4700\Omega}{3300\Omega}\right)}{1 + \frac{4700\Omega}{10000\Omega} + \frac{4700\Omega}{3300\Omega}}\\
            V_{R_1} &= 8,3760V
          \end{align*}
        \end{minipage}
        \centering
        \begin{minipage}{0.4\textwidth}
          \begin{align*}
            V_{R_2} &= V_{R_3}\\
            V_{R_3} &= V_s - V_{R_1}\\
            V_{R_3} &= 10V - 8,3760V\\
            V_{R_3} &= 1,6239V
          \end{align*}
        \end{minipage}
      \end{figure}

      Ahora, con la ley de Ohm, podemos calcular las corrientes que pasan por los resistores:
      \begin{figure}[!h]
      \centering
      \begin{minipage}{0.3\textwidth}
        \begin{align*}
          I_{R_1} &= I_t\\
          I_t &= \frac{V_{R_1}}{R_1}\\
          I_t &= \frac{8,3760V}{10000\Omega}\\
          I_t &= 837,6\mu A
        \end{align*}
      \end{minipage}
      \centering
      \begin{minipage}{0.3\textwidth}
        \begin{align*}
          I_{R_2} &= I_2\\
          I_2 &= \frac{V_{R_2}}{R_2}\\
          I_2 &= \frac{1,6239V}{4700\Omega}\\
          I_2 &= 345,5\mu A
        \end{align*}
      \end{minipage}
      \centering
      \begin{minipage}{0.3\textwidth}
        \begin{align*}
          I_{R_3} &= I_3\\
          I_3 &= \frac{V_{R_3}}{R_3}\\
          I_3 &= \frac{1,6239V}{3300\Omega}\\
          I_3 &= 492\mu A
        \end{align*}
      \end{minipage}
      \end{figure}
      
      \section{Simulación}
      Para la simulación, se utilizo el software LTSpice, en el cual se modelo el circuito, y utilizando el análisis de
      punto de operación (op ó operating point) podemos obtener las corrientes de los resistores y la caída de
      tensión del paralelo de resistores:
      \begin{figure}[!h]
        \centering
        \includegraphics[width=0.8\linewidth]{images/sim_dc.png}

        \caption{Simulación de punto de operación del circuito.}
      \end{figure}

      \section{Implementación y Mediciones}
      \begin{figure}[!h]
        \centering
        \includegraphics[angle=270, width=0.8\textwidth]{pictures/prot-crkt.jpg}
        \caption{Implementación del circuito.}
        \label{prot-crkt}
      \end{figure}
      \begin{wrapfigure}{r}{0.3\textwidth}
        \vspace{-0.3cm}
        \centering
        \includegraphics[width=0.25\textwidth]{pictures/mult-vs.jpeg}
        \caption{Calibración de la Fuente.}
        \label{mult-vs}
      \end{wrapfigure}
      Para la implementación del circuito utilizamos una Protoboard, en conjunto con cables unifilares para hacer 
      las conexiones, y resistores de carbono de $\tfrac{1}{4}W$. El circuito quedo como se ve en la figura 
      \ref{prot-crkt}.

      Seguido al ensamblaje del circuito, procedimos a conectar la fuente y calibrar la salida de voltaje lo mas
      próximo a 10V usando el multimetro, como se ve en la figura \ref{mult-vs}. La medición exacta es $9,98V$.

      Después de haber calibrado la fuente, procedimos a conectar los cables banana-cocodrilo al circuito ensamblado
      en la Protoboard y comenzar con las mediciones. A continuación se muestran una serie de imágenes en las que por
      un lado se muestra la medición del multimetro y por otro lado se muestra donde estaban tocando las puntas del
      multimetro.
      \begin{figure}[H]
        \centering
          \begin{minipage}{0.3\textwidth}
            \centering
            \includegraphics[width=1\linewidth]{pictures/mult-v_r1.jpg}
            \includegraphics[width=1\linewidth]{pictures/prot-v_r1.jpg}
            \caption{Medición $V_{R_1}$.}
          \end{minipage}
          \begin{minipage}{0.3\textwidth}
            \centering
            \includegraphics[width=1\linewidth]{pictures/mult-v_r2-r3.jpg}
            \includegraphics[width=1\linewidth]{pictures/prot-v_r2-r3.jpg}
            \caption{Medición $V_{R_2}$.}
          \end{minipage}
          \begin{minipage}{0.3\textwidth}
            \centering
            \includegraphics[width=1\linewidth]{pictures/mult-it.jpg}
            \includegraphics[width=1\linewidth]{pictures/prot-it.jpg}
            \caption{Medición $I_t$.}
          \end{minipage}
      \end{figure}
      \vspace{-0.5cm}
      \begin{figure}[H]
        \centering
          \begin{minipage}{0.3\textwidth}
            \centering
            \includegraphics[width=1\linewidth]{pictures/mult-i2.jpg}
            \includegraphics[width=1\linewidth]{pictures/prot-i2.jpg}
            \caption{Medición $I_2$.}
          \end{minipage}
          \begin{minipage}{0.3\textwidth}
            \centering
            \includegraphics[width=1\linewidth]{pictures/mult-i3.jpg}
            \includegraphics[width=1\linewidth]{pictures/prot-i3.jpg}
            \caption{Medición $I_3$.}
          \end{minipage}
      \end{figure}


      Con el objetivo de comparar, se presenta a continuación una tabla de los valores calculados con resistores y
      fuentes ideales y los valores medidos en la practica de laboratorio:
      \begin{figure}[!h]
        \centering
        \begin{tabular}[c]{|c||c|c|c|c||c|}
          \hline
          \multicolumn{6}{|c|}{Comparativa de valores}\\
          \hline
                    & $V_s$         & $R_1$         & $R_2$         & $R_3$         & \\
          \hline
          Tensión   & $10V$         & $8,3760V$     & \multicolumn{2}{|c||}{$1,6239V$} & \multirow{2}{*}{Ideal}\\
          \cline{1-5}
          Corriente &               & $837,6\mu A$  & $345,5\mu A$  & $492\mu A$    & \\
          \hline
          Tensión   & $9,98V$       & $8,36V$       & \multicolumn{2}{|c||}{$1,611V$}  & \multirow{2}{*}{Real}\\
          \cline{1-5}
          Corriente &               & $829\mu A$    & $340\mu A$    & $481\mu A$    & \\
          \hline
        \end{tabular}
      \end{figure}
      
      La precisión de los valores de la realidad con respecto a los ideales tiene que ver con las resistencias de
      contacto al colocar los elementos en la Protoboard y las puntas del multimetro, y las tolerancias de fabricación
      de los resistores (en este caso, 5\%).

    \myemptypage
    \chapter{Actividad Anexa}
      El profesor propuso, a modo de interiorizarse con las herramientas de trabajo del laboratorio, reemplazar la
      fuente de corriente continua por un generador de funciones. Este mismo, configurado a $10Vpp$, onda sinusoidal y
      frecuencia de $50Hz$ simula un transformador de corriente alterna. De esta forma, podemos emplear el Osciloscopio
      para poder ver como se comportan los resistores bajo corrientes alternas.

      \section{Simulación}
      Para la simulación utilizamos nuevamente el software LTSpice, esta vez con el comando spice .tran (transient
      analysis) con $0.04s$ de muestra para ver dos ciclos de la señal alterna:
      \begin{figure}[!h]
        \centering
        \includegraphics[width=1\linewidth]{images/sim_ac.png}
        \caption{Simulación en corriente alterna.}
      \end{figure}

      Podemos entonces observar que la caída de tensión pico-pico en el resistor $R_1$ es de aproximadamente $8,376V$,
      y en el resistor $R_2$ es de aproximadamente $1,623V$. Así mismo, las corrientes tienen valores pico-pico
      similares a los calculados en corriente continua. Para $i_t$ son $837,6\mu A$, para $i_2$, $345,4\mu A$ y para
      $i_3$, $491,9\mu A$.

      \newpage
      \section{Mediciones}
      Configurando el generador de funciones, y después de haber calibrado el osciloscopio, podemos ver la siguiente
      señal en el mismo:
      \begin{figure}[!h]
        \centering
        \includegraphics[width=0.8\linewidth]{pictures/osc-ac_vs.jpeg}
        \caption{Calibración del generador de funciones. Osciloscopio configurado a 2V/div, 5mS/div.}
      \end{figure}

      Si conectamos ahora la salida del generador de funciones a nuestro circuito, podemos observar las caídas de
      tensión en cada resistor:
      \begin{figure}[!h]
        \centering
        \begin{minipage}{0.45\textwidth}
          \includegraphics[width=1\linewidth]{pictures/osc-mult-ac_v_r1.jpeg}
          \caption{$V_{R_1}$. Osciloscopio configurado a 1V/div, 5mS/div. Lectura: $3,54V$}
        \end{minipage}
        \hspace{0.5cm}
        \begin{minipage}{0.45\textwidth}
          \includegraphics[width=1\linewidth]{pictures/osc-ac-mult-v_r2-r3.jpeg}
          \caption{$V_{R_2}$. Osciloscopio configurado a 1V/div, 5mS/div. Lectura: $0,576V$}
        \end{minipage}
      \end{figure}

      Naturalmente, los números no coinciden, esto es debido a que el multimetro esta midiendo el valor eficaz de la
      caída de tensión en las resistencias, y no el valor pico o el valor pico-pico. Así y todo, al momento de escribir
      el informe, quisimos corroborar esas mediciones, y nos dimos cuenta que posiblemente hayamos calibrado mal el
      osciloscopio, y por lo tanto el generador de funciones, ya que los valores eficaces, pasados a valores pico-pico,
      no se acercan por algunos volts a los valores simulados.

\end{document}
