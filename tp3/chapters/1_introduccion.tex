\chapter{Introduccion}

En este trabajo práctico se aborda el estudio del transistor bipolar de juntura (BJT), un dispositivo fundamental en electrónica analógica y digital. El BJT se compone de tres regiones semiconductoras —emisor, base y colector— separadas por dos uniones PN. Estas permiten que el dispositivo opere tanto como amplificador de señal como en funciones de conmutación. Su funcionamiento se basa en el transporte de portadores de carga mayoritarios y minoritarios (electrones y huecos), lo que le da el nombre de "bipolar".\\
El objetivo de este trabajo es comprender el comportamiento del transistor en sus diferentes regiones de operación: corte, activa y saturación. Se estudiará el efecto de la polarización de sus uniones, se obtendrán curvas características a través de simulaciones y mediciones prácticas, y se analizará la ganancia de corriente $\beta$, fundamental para su aplicación como amplificador.\\
La metodología incluye tanto la implementación de circuitos en plataformas de simulación como la medición directa en laboratorio. Esto permitirá comparar resultados teóricos, simulados y experimentales, desarrollando criterios sólidos de análisis. También se analizarán especificaciones reales de distintos transistores a partir de sus hojas de datos, reforzando la lectura técnica y la interpretación de parámetros como $V_{CE}(sat), h_{FE}(max)$ y disipacion de potencia.\\
Este trabajo práctico no solo permite afianzar los conocimientos sobre dispositivos semiconductores, sino que también nos prepara para su correcta aplicación en diseños electrónicos reales, desarrollando habilidades analíticas, experimentales y de documentación técnica.
