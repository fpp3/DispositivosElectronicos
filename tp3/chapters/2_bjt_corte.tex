\chapter{Transistor en Zona de Corte}
  La \textbf{zona de corte} es una de las regiones fundamentales de operación del transistor bipolar (BJT). En esta
  condición, ambas uniones PN (base-emisor y base-colector) están polarizadas en inversa, impidiendo el flujo de
  corriente significativa a través del dispositivo. Como consecuencia, tanto la corriente de colector $I_C$ como la
  corriente de emisor $I_E$ son prácticamente nulas.

  El BJT en corte se comporta como un \textit{interruptor abierto}, motivo por el cual esta región se utiliza
  ampliamente en aplicaciones de conmutación digital, como puertas lógicas o etapas de control.

  Para que el transistor esté en corte, se debe cumplir:
  \begin{equation}
  V_{BE} < V_{BE(\text{on})}
  \end{equation}
  donde $V_{BE(\text{on})}$ representa la tensión de umbral para el encendido del transistor (típicamente entre 0.6\,V y
  0.7\,V para transistores de silicio).

  En esta región de operación se verifica que:
  \begin{itemize}
    \item $I_B \approx 0$
    \item $I_C \approx 0$
    \item $V_{CE} \approx V_{CC}$
  \end{itemize}

  Durante el trabajo práctico, se verificará experimentalmente esta región observando que, al no aplicarse una tensión
  suficiente en la base, el transistor permanece ''apagado'' y no conduce corriente, confirmando así su comportamiento
  en corte.

  \section{Simulacion}
    En la figura (\ref{crkt:corte}) se presenta el circuito utilizado en la simulacion y en el listado
    (\ref{list:corte}) los parametros que se utilizaron para obtener el $I_{CEO}$.

    \begin{figure}[!ht]
      \centering
      \begin{minipage}{0.45\textwidth}
        \begin{tikzpicture}[circuitikz, straight voltages]
          % Paths, nodes and wires:
          \draw node[npn] at (8, 6) {};
          \draw (7.16, 6) to[american resistor, l={$R_B$}, label distance=0.02cm] (5, 6);
          \draw (9, 7.75) to[american resistor, l={$R_C$}, label distance=0.02cm, v<=$I_{CEO}$] (11, 7.75);
          \draw (8, 6.52) -| (8, 7.75);
          \draw (12, 7) to[battery, l={$V_{CC}$}, label distance=0.02cm] (12, 6);
          \draw (12, 6.75) -| (12, 7.75);
          \draw (8, 5.23) -| (8, 4.5);
          \draw (12, 6.011) |- (12, 4.5);
          \draw node[ground] at (8, 4.5) {};
          \draw node[ground] at (12, 4.5) {};
          \draw (8, 7.75) -- (9, 7.75);
          \draw (11, 7.75) -- (12, 7.75);
          \draw node[ground] at (5, 4.5) {};
          \draw (5, 6) -| (5, 4.5);
        \end{tikzpicture}
        \caption{Circuito de prueba para polarizacion en zona de corte.}
        \label{crkt:corte}
      \end{minipage}
      \hfill
      \begin{minipage}{0.45\textwidth}
        \begin{lstlisting}[style=ltspice, caption={Parámetros de simulación LTspice}, label=list:corte]
.model BC548C   NPN(Is=7.049f Xti=3 Eg=1.11 Vaf=54.76 Bf=543.1 Ise=78.17f Ne=1.479 Ikf=24.96m Nk=.5381 Xtb=1.5 Br=1.2 Isc=27.51f Nc=1.775 Ikr=3.321 Rc=.9706 Cjc=4.25p Mjc=.3147 Vjc=.5697 Fc=.5 Cje=11.5p Mje=.6715 Vje=.5 Tr=10n Tf=410.7p Itf=1.12 Xtf=26.19 Vtf=10)
.op
        \end{lstlisting}
      \end{minipage}
    \end{figure}

    La simulacion de punto de operacion dio como resultado una $I_C = 10pA$. Por otro lado, el
    valor aproximado que brinda el fabricante, tiene que ser calculado, ya que solamente presenta el valor de $I_{CBO}$.

    Con los datos brindados por el fabricante en el datahseet:
    \begin{itemize}
      \item $I_{CBO}=15 \, nA \, @ \, V_{CB} = 30V$
      \item $\beta$ o $h_{fe}max = 450$
    \end{itemize}

    Teniendo el $\beta$ calculo el $\alpha$ con la siguiente formular:
    \begin{gather*}
      \alpha=\frac{\beta}{\beta+1}\\
      \alpha=0.9978
    \end{gather*}
    Luego, se puede obtener la $I_{CEO}$ con la siguiente formula:
    \begin{equation}
      I_{CEO}=\frac{I_{CBO}}{1-\alpha}
    \end{equation}
    \begin{equation*}
      I_{CEO}= 6.75 \mu A
    \end{equation*}

    El valor calculado difiere con el de la simulacion, debido a que el modelo posee valores idealizados del transistor,
    ademas de que tiene otros datos (como corrientes de fuga de colector y emisor) que en el componente real serian
    imposibles de calcular o medir debido a su magnitud (en el orden de los fA).

    Para finalizar respondiendo la pregunta "¿Cree poder implementar este circuito en el laboratorio y realizar la
    medición de la corriente de fuga?":

    Sí, es posible implementar el circuito en el laboratorio, aunque con algunas consideraciones importantes. La
    corriente de fuga $I_{CEO}$ en un transistor bipolar es muy pequeña (del orden de nanoamperios en condiciones
    normales), por lo que su medición directa requiere instrumentos con alta sensibilidad, como un amperímetro o
    multímetro con resolución en el rango de nA.

    Para que la medición sea significativa:

    \begin{itemize}
      \item Es necesario mantener la base abierta o conectada mediante una resistencia de valor muy alto (simulando
          circuito abierto).
      \item Se debe aplicar una tensión $V_{CE}$ adecuada (por ejemplo, entre 5\,V y 10\,V).
      \item Puede ser útil aumentar la temperatura del transistor (por ejemplo, con una pistola de aire caliente) para
          incrementar $I_{CEO}$, ya que esta corriente crece exponencialmente con la temperatura.
    \end{itemize}

    En condiciones estándar de laboratorio, puede ser difícil distinguir $I_{CEO}$ de las corrientes de fuga del
    instrumento o del ruido eléctrico, pero con una buena configuración experimental, la medición es viable y permite
    confirmar el comportamiento esperado del transistor en zona de corte.
