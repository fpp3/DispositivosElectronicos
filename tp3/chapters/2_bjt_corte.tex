\chapter{Transistor en Zona de Corte}
    La \textbf{zona de corte} es una de las regiones fundamentales de operación del transistor bipolar (BJT). En esta condición, ambas uniones PN (base-emisor y base-colector) están polarizadas en inversa, impidiendo el flujo de corriente significativa a través del dispositivo. Como consecuencia, tanto la corriente de colector $I_C$ como la corriente de emisor $I_E$ son prácticamente nulas.

    El BJT en corte se comporta como un \textit{interruptor abierto}, motivo por el cual esta región se utiliza ampliamente en aplicaciones de conmutación digital, como puertas lógicas o etapas de control.
    
    Para que el transistor esté en corte, se debe cumplir:
    
    \[
    V_{BE} < V_{BE(\text{on})}
    \]
    
    donde $V_{BE(\text{on})}$ representa la tensión de umbral para el encendido del transistor (típicamente entre 0.6\,V y 0.7\,V para transistores de silicio).
    
    En esta región de operación se verifica que:
    
    \begin{itemize}
      \item $I_B \approx 0$
      \item $I_C \approx 0$
      \item $V_{CE} \approx V_{CC}$
    \end{itemize}
    
    Durante el trabajo práctico, se verificará experimentalmente esta región observando que, al no aplicarse una tensión suficiente en la base, el transistor permanece ``apagado'' y no conduce corriente, confirmando así su comportamiento en corte.


\newpage


  \section{Simulacion}


  A continuacion se presenta el circuito utilizado en la simulacion y los parametros que se utilizaron para conseguir la grafica con los datos requeridos:



    %Circuito de la simulacion

    %Grafico de la simulacion


  La grafica con los valores obtenidos de la $I_{CEO}$ que presenta la simulacion es la siguiente:\\


  Como se puede observar los valores estan cerca de los $10pA$ siendo este un valor muy pequeño, por otro lado, el valor aproximado que me puede brindar el datasheet tiene que ser calculado ya que solamente me da el $I_{CBO}$.


  Datos brindados por datasheet:

  \begin{itemize}
      \item $I_{CBO}=15nA$
      \item $\beta$ o $h_{fe}max=450$
  \end{itemize}

  Teniendo el $\beta$ calculo el $\alpha$ con la siguiente formular:


    
    \begin{center}

        \[
        \alpha=\frac{\beta}{\beta+1}
        \]
        \[
        \alpha=0.9978
        \]
        
    \end{center}


    Luego calculo la $I_{CEO}$ con la siguiente formula:


    \begin{center}
        \[
        I_{CEO}=\frac{I_{CBO}}{1-\alpha}
        \]

        \[
        I_{CEO}= 6.81\mu A
        \]
           
    \end{center}

  Para finalizar respondiendo la pregunta "¿Cree poder implementar este circuito en el laboratorio y
realizar la medición de la corriente de fuga?":

    Sí, es posible implementar el circuito en el laboratorio, aunque con algunas consideraciones importantes. La corriente de fuga $I_{CEO}$ en un transistor bipolar es muy pequeña (del orden de nanoamperios en condiciones normales), por lo que su medición directa requiere instrumentos con alta sensibilidad, como un amperímetro o multímetro con resolución en el rango de nA.
    
    Para que la medición sea significativa:
    
    \begin{itemize}
        \item Es necesario mantener la base abierta o conectada mediante una resistencia de valor muy alto (simulando circuito abierto).
        \item Se debe aplicar una tensión $V_{CE}$ adecuada (por ejemplo, entre 5\,V y 10\,V).
        \item Puede ser útil aumentar la temperatura del transistor (por ejemplo, con una pistola de aire caliente) para incrementar $I_{CEO}$, ya que esta corriente crece exponencialmente con la temperatura.
    \end{itemize}
    
    En condiciones estándar de laboratorio, puede ser difícil distinguir $I_{CEO}$ de las corrientes de fuga del instrumento o del ruido eléctrico, pero con una buena configuración experimental, la medición es viable y permite confirmar el comportamiento esperado del transistor en zona de corte.


       
       
        
