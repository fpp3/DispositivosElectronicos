\chapter{Interpretacion de Hojas de datos}

  \begin{itemize}
    \item Caracteristicas Electricas a 25°C:
      \begin{enumerate}
        \item \textbf{$V_{(BR)CEO}$} — Tensión de ruptura colector-emisor con base abierta: \\
        Tensión máxima que puede aplicarse entre colector y emisor con la base abierta sin que ocurra ruptura. Si se
              supera, puede dañar permanentemente el transistor.
        \item \textbf{$I_{CEO}$} — Corriente de fuga colector-emisor con base abierta: \\
        Corriente que circula desde el colector al emisor cuando la base está abierta. Es una corriente muy pequeña (en
              el orden de nA a µA) causada por portadores minoritarios.
        \item \textbf{$I_{CES}$} — Corriente de fuga colector-emisor con base conectada al emisor: \\
        Similar a $I_{CEO}$, pero en este caso la base está conectada directamente al emisor. Es aún menor y representa
              el nivel mínimo de fuga entre colector y emisor.
        \item \textbf{$I_C$} — Corriente máxima de colector: \\
        Es la máxima corriente que puede circular de forma continua por el colector sin dañar el dispositivo. Se
              relaciona con la disipación máxima de potencia.
        \item \textbf{$I_{EBO}$} — Corriente de fuga emisor-base con colector abierto: \\
        Corriente inversa entre emisor y base cuando el colector está en circuito abierto. También muy baja; mide la
              fuga a través de la juntura BE.      
      \end{enumerate}
    \item Caracteristicas Termicas:
      \begin{enumerate}
        \item \textbf{$\theta_{jc}$} — Resistencia térmica junta-colector: \\
        Indica la resistencia térmica entre la juntura interna del transistor y el colector (normalmente encapsulado).
              Se mide en °C/W. Cuanto menor, mejor disipación.
        \item \textbf{$\theta_{ja}$} — Resistencia térmica junta-ambiente: \\
        Representa la resistencia térmica total desde la juntura hasta el ambiente. Permite estimar cuánto se calentará
              el transistor por cada watt disipado.
      \end{enumerate}
    \item Caracteristicas de conmutacion a 25°C:
      \begin{enumerate}
        \item \textbf{$t_{on}$} — Tiempo de encendido: \\
        Tiempo que tarda el transistor en pasar del estado de corte al estado de conducción cuando se le aplica una
              señal.
        \item \textbf{$h_{FE}$} — Ganancia de corriente continua (DC): \\
        Relación entre la corriente de colector y la corriente de base ($I_C / I_B$) en continua. También conocida como
              $\beta$.
        \item \textbf{$h_{fe}$} — Ganancia de corriente en pequeña señal (AC): \\
        Relación $I_C / I_B$ medida con señales pequeñas de alterna. Se utiliza para análisis de pequeñas señales.
        \item \textbf{$V_{BE}$} — Tensión base-emisor: \\
        Tensión directa entre base y emisor necesaria para que el transistor comience a conducir. Para silicio, suele
              estar entre 0.6\,V y 0.7\,V.
        \item \textbf{$V_{CE(sat)}$} — Tensión de saturación colector-emisor: \\
        Tensión entre colector y emisor cuando el transistor está completamente saturado. Suele ser baja $(\approx 0.2\,V)$,
              lo que indica buen contacto entre colector y emisor.
        \item \textbf{$P_d$ o $P_{tot}$} — Potencia máxima de disipación: \\
        Potencia máxima que puede disipar el transistor sin superar su temperatura máxima de operación. Se calcula como:
        \[
        P_{d} = \frac{T_{\text{max}} - T_{\text{amb}}}{\theta_{ja}}
        \]
      \end{enumerate}
  \end{itemize}


\begin{table}[!ht]
\centering
\resizebox{\textwidth}{!}{%
\begin{tabular}{|c|c|c|c|c|c|c|}
\hline
\textbf{Parámetro} & \textbf{BC548} & \textbf{BC557} & \textbf{2N2222} & \textbf{BU208} & \textbf{MPS6514} & \textbf{TIP36} \\
\hline
$V_{(BR)CEO}$  & $30\,V$ & $45\,V$ & NO PRESENTA & NO PRESENTA & $25\,V$ & $40\,V$ \\
\hline
$I_{CEO}$      & NO PRESENTA & NO PRESENTA & NO PRESENTA & NO PRESENTA & NO PRESENTA & $1\,mA$ \\
\hline
$I_{CES}$      & $0.2\,nA$ & $100\,nA$ & NO PRESENTA & $<1\,mA$ & NO PRESENTA & $0.7\,mA$ \\
\hline
$I_C$          & $500\,mA$ & $100\,mA$ & $800\,mA$ & $<5\,A$ & $200\,mA$ & $25\,A$ \\
\hline
$I_{EBO}$      & NO PRESENTA & NO PRESENTA & $10\,nA$ & NO PRESENTA & NO PRESENTA & $1\,mA$ \\
\hline
$\theta_{jc}$  & $83.3\,^\circ$C/W & $83.3\,^\circ$C/W & $146\,K/W$ & $2.5\,^\circ$C/W & $83.3\,^\circ$C/W & $1\,^\circ$C/W \\
\hline
$\theta_{ja}$  & $200\,^\circ$C/W & $200\,^\circ$C/W & $350\,K/W$ & NO PRESENTA & $200\,^\circ$C/W & $200\,^\circ$C/W \\
\hline
$t_{on}$       & NO PRESENTA & NO PRESENTA & $35\,\mathrm{ns}$ & NO PRESENTA & NO PRESENTA & $1.1\,\mu S$ \\
\hline
$h_{FE}$       & 800 & 800 & 300 & $>2.25$ & 300 & 75 \\
\hline
$h_{fe}$       & 900 & 900 & NO PRESENTA & NO PRESENTA & NO PRESENTA & 25 \\
\hline
$V_{BE}$       & $0.70\,V$ & $0.7\,V$ & NO PRESENTA & NO PRESENTA & NO PRESENTA & $2\,V$ \\
\hline
$V_{CE(sat)}$  & $0.25\,V$ & $0.3\,V$ & $300\,mV$ & $1\,V$ & $0.5\,V$ & $1.8\,V$ \\
\hline
$P_d$          & $625\,\mathrm{mW}$ & $625\,\mathrm{mW}$ & $500\,\mathrm{mW}$ & $<1.25\,\mathrm{W}$ & $625\,\mathrm{mW}$ & $125\,\mathrm{W}$ \\
\hline
\end{tabular}
}
\caption{Comparación de parámetros eléctricos, térmicos y de conmutación entre distintos transistores BJT.}
\end{table}
