\chapter{Conclusiones}
  A lo largo de este trabajo práctico se logró un abordaje integral del transistor bipolar de juntura (BJT), mediante el
  análisis teórico, la simulación y la experimentación en laboratorio. Se estudiaron detalladamente las tres regiones de
  operación —corte, activa y saturación— validando el comportamiento del dispositivo en cada una de ellas. Las curvas
  características obtenidas permitieron visualizar con claridad las relaciones entre corrientes y tensiones, y cómo
  éstas dependen de la polarización del transistor.
  
  La metodología combinada, que incluyó simulaciones con LTSpice y prácticas en laboratorio, permitió no solo contrastar
  los modelos teóricos con la realidad, sino también comprender las limitaciones prácticas de medición, como la
  dificultad para registrar corrientes de fuga extremadamente pequeñas. A su vez, se detectaron errores metodológicos en
  la toma de datos, lo que resalta la importancia de una planificación experimental rigurosa y la necesidad de mantener
  condiciones controladas para obtener resultados válidos.
  
  Además, el análisis de hojas de datos de distintos transistores fortaleció la lectura técnica, permitiendo interpretar
  parámetros clave como $h_{FE}$, $V_{CE(\text{sat})}$, $P_d$ y resistencias térmicas. Esto brinda herramientas
  fundamentales para la correcta selección de componentes en futuros diseños electrónicos.
  
  En síntesis, este trabajo permitió consolidar conocimientos fundamentales sobre el funcionamiento del BJT, su
  caracterización eléctrica y su aplicación en circuitos. También fortaleció habilidades prácticas, de análisis crítico
  y documentación técnica, esenciales en la formación de un profesional en electrónica.
