\chapter{Conclusiones}

A lo largo de este trabajo práctico se abordaron en profundidad los principios fundamentales de funcionamiento del transistor bipolar de juntura (BJT). Mediante distintas experiencias experimentales, simulaciones y análisis teóricos, se caracterizaron sus principales regiones de operación: corte, activa y saturación.

En primer lugar, se verificó experimentalmente el comportamiento de la \textbf{juntura base-emisor}, observando su conducción al superar los 0.6--0.7\,V, característico de un diodo de silicio. Luego se midieron las \textbf{curvas características} $I_C$ vs.\ $V_{CE}$ para distintos valores de corriente de base, identificando con claridad las transiciones entre las regiones de corte, activa y saturación. Estos resultados permitieron visualizar cómo se comporta el transistor como amplificador y como conmutador, dependiendo de las condiciones de polarización.

Asimismo, se determinaron las \textbf{características de transferencia de corriente}, es decir, la relación $I_C$ vs.\ $I_B$, a partir de la cual se estimó la ganancia de corriente continua $\beta$ del dispositivo. Se verificó que $\beta$ se mantiene aproximadamente constante en la región activa, aunque puede variar levemente con la corriente o temperatura, lo cual es coherente con las hojas de datos del fabricante.

Las simulaciones realizadas complementaron los ensayos prácticos, permitiendo obtener curvas más precisas y confirmar el comportamiento ideal teórico frente a un modelo físico más realista. Además, el análisis del \textit{efecto Early} y la interpretación de parámetros como $V_{CE(\text{sat})}$ o $h_{FE}$ en el \textit{datasheet} permitieron afianzar el vínculo entre teoría y aplicación.

En conclusión, este trabajo permitió no solo comprender el modelo funcional del BJT, sino también desarrollar criterios de análisis experimental, interpretación de resultados y lectura de documentación técnica, fundamentales para su aplicación en circuitos reales tanto de potencia como de señal.
