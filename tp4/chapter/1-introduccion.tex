\chapter{Introducción}
El presente trabajo práctico tiene como objetivo el estudio del transistor de efecto de campo de juntura (JFET), un
dispositivo semiconductor ampliamente utilizado en aplicaciones de conmutación y amplificación debido a su alta
impedancia de entrada y bajo nivel de ruido. A lo largo del desarrollo se analizan sus distintas regiones de operación
—corte, zona óhmica y saturación— tanto desde la simulación en LTSpice como mediante experiencias de laboratorio.

Mediante la caracterización experimental y teórica se busca comprender cómo la tensión aplicada entre compuerta y fuente
($V_{GS}$) controla la corriente de drenaje ($I_D$), así como verificar la validez de los modelos matemáticos, en
particular la ecuación de Shockley. Además, se interpretan parámetros clave a partir de la hoja de datos del componente,
lo que permite relacionar la teoría con el comportamiento real del dispositivo.

Este trabajo integra simulación, práctica de laboratorio y análisis de datos, fomentando una visión integral del
funcionamiento del JFET y reforzando los conceptos fundamentales de los dispositivos electrónicos.
