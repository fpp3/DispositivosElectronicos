\chapter{Introducción}
En el presente trabajo práctico se estudian las características eléctricas del transistor de efecto de campo de juntura
(JFET), dispositivo ampliamente utilizado en aplicaciones analógicas por su alta impedancia de entrada y su capacidad de
control mediante voltaje. 

El objetivo principal es comprender el comportamiento del JFET en sus distintas regiones de operación (saturación y
corte) a partir de simulaciones realizadas en LTSpice y de experiencias de laboratorio con el transistor 2N5457. 

A lo largo del informe se analizan las curvas características de salida y de transferencia, así como los parámetros
fundamentales del dispositivo, tales como la corriente de saturación $I_{DSS}$ y la tensión de corte $V_{GS(off)}$.
Finalmente, se realiza una comparación entre los resultados experimentales, los obtenidos por simulación y los datos
proporcionados en la hoja de datos del fabricante, con el fin de evaluar la precisión de los modelos y afianzar los
conceptos teóricos desarrollados en clase.
