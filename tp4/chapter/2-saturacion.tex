\chapter{JFET en Saturación}
  \begin{wrapfigure}{R}{0.3\textwidth}
  \vspace{-1cm}
    \centering
    \resizebox{!}{\linewidth}{
    \begin{tikzpicture}
	\begin{pgfonlayer}{nodelayer}
		\node [style=none] (0) at (-2, 3.25) {};
		\node [style=none] (1) at (-2, -3.25) {};
		\node [style=none] (2) at (2, -3.25) {};
		\node [style=none] (3) at (2, 3.25) {};
		\node [style=none] (4) at (-2, 1.25) {};
		\node [style=none] (6) at (-1.25, 1.25) {};
		\node [style=none] (7) at (-1.25, -1.25) {};
		\node [style=none] (8) at (1.25, 1.25) {};
		\node [style=none] (9) at (1.25, -1.25) {};
		\node [style=none] (10) at (2, 1.25) {};
		\node [style=none] (11) at (2, -1.25) {};
		\node [style=none] (12) at (-2, -1.25) {};
		\node [style=none] (13) at (-2, 1.75) {};
		\node [style=none] (14) at (-0.75, 1.75) {};
		\node [style=none] (15) at (-0.75, -1.75) {};
		\node [style=none] (16) at (-2, -1.75) {};
		\node [style=none] (17) at (2, 1.75) {};
		\node [style=none] (18) at (0.75, 1.75) {};
		\node [style=none] (19) at (0.75, -1.75) {};
		\node [style=none] (20) at (2, -1.75) {};
		\node [style=none] (21) at (0, 0) {$n$};
		\node [style=none] (22) at (-1.5, 0) {$p$};
		\node [style=none] (23) at (1.5, 0) {$p$};
		\node [style=none] (24) at (-0.5, 3.25) {};
		\node [style=none] (25) at (-0.5, 3.5) {};
		\node [style=none] (26) at (0.5, 3.5) {};
		\node [style=none] (27) at (0.5, 3.25) {};
		\node [style=none] (28) at (-0.5, -3.25) {};
		\node [style=none] (29) at (-0.5, -3.5) {};
		\node [style=none] (30) at (0.5, -3.5) {};
		\node [style=none] (31) at (0.5, -3.25) {};
		\node [style=none] (32) at (-2, 0.5) {};
		\node [style=none] (35) at (-2.25, 0.5) {};
		\node [style=none] (36) at (2, 0.5) {};
		\node [style=none] (37) at (2, -0.5) {};
		\node [style=none] (38) at (2.25, -0.5) {};
		\node [style=none] (39) at (2.25, 0.5) {};
		\node [style=none] (40) at (-2.25, -0.5) {};
		\node [style=none] (41) at (-2, -0.5) {};
		\node [style=none] (42) at (-2.25, 0) {};
		\node [style=none] (43) at (-3.5, 0) {};
		\node [style=none] (44) at (2.25, 0) {};
		\node [style=none] (45) at (2.75, 0) {};
		\node [style=none] (46) at (2.75, 2.5) {};
		\node [style=none] (47) at (2, 2.5) {};
		\node [style=none] (48) at (-2, 2.5) {};
		\node [style=none] (49) at (-2.75, 2.5) {};
		\node [style=none] (50) at (0, 3.5) {};
		\node [style=none] (51) at (0, 4) {};
		\node [style=none] (52) at (0, -3.5) {};
		\node [style=none] (53) at (0, -4) {};
		\node [style=small dot] (54) at (-2.75, 0) {};
		\node [style=none] (55) at (0, 4.5) {D};
		\node [style=none] (56) at (0, -4.5) {S};
		\node [style=none] (57) at (-4, 0) {G};
	\end{pgfonlayer}
	\begin{pgfonlayer}{edgelayer}
		\draw [style=fill1] (3.center)
			 to [in=90, out=-90] (2.center)
			 to (1.center)
			 to (0.center)
			 to cycle;
		\draw [style=fill2] (7.center)
			 to (12.center)
			 to (4.center)
			 to (6.center)
			 to cycle;
		\draw [style=fill2] (8.center)
			 to (10.center)
			 to (11.center)
			 to (9.center)
			 to cycle;
		\draw [style=fill3] (6.center)
			 to (4.center)
			 to (13.center)
			 to (14.center)
			 to (15.center)
			 to (16.center)
			 to (12.center)
			 to (7.center)
			 to cycle;
		\draw [style=fill3] (8.center)
			 to (10.center)
			 to (17.center)
			 to (18.center)
			 to (19.center)
			 to (20.center)
			 to (11.center)
			 to (9.center)
			 to cycle;
		\draw [style=fill4] (32.center)
			 to (35.center)
			 to (40.center)
			 to (41.center)
			 to cycle;
		\draw [style=fill4] (39.center)
			 to (36.center)
			 to (37.center)
			 to (38.center)
			 to cycle;
		\draw [style=fill4] (25.center)
			 to (26.center)
			 to (27.center)
			 to (24.center)
			 to cycle;
		\draw [style=fill4] (29.center)
			 to (30.center)
			 to (31.center)
			 to (28.center)
			 to cycle;
		\draw (50.center) to (51.center);
		\draw (44.center) to (45.center);
		\draw (45.center) to (46.center);
		\draw (46.center) to (47.center);
		\draw (48.center) to (49.center);
		\draw (42.center) to (43.center);
		\draw (52.center) to (53.center);
		\draw (54) to (49.center);
		\draw [style=dashedd] (47.center) to (48.center);
	\end{pgfonlayer}
\end{tikzpicture}

    }
    \caption{estructura interna del JFET.}
    \label{fig:jfet.nopol}
  \end{wrapfigure}
  Si al JFET se le aplica un voltaje positivo $V_{DS}$ a través del canal y el gate está conectada directamente a la
  fuente, estableciendo la condición de $V_{GS} = 0 V$, el resultado es un gate y un source al mismo potencial y una
  región de empobrecimiento en el extremo bajo de cada material p similar a la distribución de las condiciones sin
  polarización, como se puede ver en la figura \ref{fig:jfet.nopol}. La región gris oscuro simboliza la región de
  empobrecimiento generada por la juntura de las regiones PN.

  En el instante en que se aplica $V_{DD} (= V_{DS})$, los electrones son atraídos hacia el drenaje y se establece la
  corriente convencional $I_D$. En este momento, el flujo de la carga está relativamente desinhibido y limitado sólo
  por la resistencia del canal n entre la fuente y el drenaje.

  \begin{wrapfigure}{l}{0.4\textwidth}
    \centering
    \resizebox{!}{\linewidth}{
    \input{tikz/jfet_pol-sat.tikz}
    }
    \caption{JFET polarizado.}
    \label{fig:jfet.pol-sat}
  \end{wrapfigure}
  Conforme el voltaje $V_{DS}$ aumente de $0V$ a algunos volts, la corriente también lo hará de acuerdo con la ley de Ohm.
  Al mismo tiempo, las regiones de empobrecimiento se iran acercando mas y mas, hasta llegar a un potencial $V_p$, en el
  cual las regiones se tocan y se produce un estrangulamiento. La palabra estrangulamiento puede llegar a interpretarse
  como que "deja de circular corriente", pero en realidad sigue existiendo un canal muy pequeño, con una corriente de
  muy alta densidad.

  A medida que $v_{DS}$ aumenta más allá de $V_p$ , la longitud de la región del encuentro cerca no entre las dos
  regiones de empobrecimiento crece a lo largo del canal, pero el nivel de $I_D$ permanece igual. Por consiguiente, una
  vez que $V_{DS} > V_p$, el JFET tiene las características de una fuente de corriente. Como se muestra en la figura
  6.9, la corriente se mantiene fija en el valor $I_D = I_{DSS}$ , pero la carga aplicada determina el voltaje $V_{DS}$
  (para niveles $\geq V_p$).

  \section{Actividad de Simulación}
    Se propuso implementar el circuito de la figura \ref{crkt:jfet-sat} en el simulador LTSpice, y hacer que la fuente
    $V1$ varíe desde $0V$ a $15V$ en pasos de $0.1V$, para poder recrear una curva que exponga el comportamiento de la
    corriente $I_D$ en función del voltaje drain-source $V_{DS}$.
    \begin{figure}[!ht]
      \centering
      \begin{minipage}{0.45\textwidth}
        \begin{tikzpicture}
          % Paths, nodes and wires:
          \node[njfet](N1) at (0, -0){} node[anchor=west] at (N1.text){$2N5457$};
          \draw (2.5, 0.75) to[battery, l={$V1$}] (2.5, -0.75);
          \draw (2.5, 0.73) -- (2.5, 2.48) |- (0, 2.48) -| (0, 0.75);
          \draw (0, -0.77) -- (0, -2);
          \draw (2.5, -0.75) -- (2.5, -2);
          \draw (-0.98, -0.27) -| (-2, -2);
          \node[ground] at (-2, -2){};
          \node[ground] at (0, -2){};
          \node[ground] at (2.5, -2){};
        \end{tikzpicture}
        \caption{circuito de prueba para saturación.}
        \label{crkt:jfet-sat}
      \end{minipage}
      \hfill
      \begin{minipage}{0.45\textwidth}
        \begin{lstlisting}[style=ltspice, caption={Parámetros de simulación LTspice}, label=list:jfet-sat]
.MODEL 2N5457 NJF IS=1N VT0=-1.5 BETA=1.125M LAMBDA=2.3M CGD=4PF CGS=5PF
.dc V1 0 15 .1
        \end{lstlisting}
      \end{minipage}
    \end{figure}

    \begin{figure}[!ht]
      \centering
      \includegraphics[width=0.8\textwidth]{images/saturacion-id_vds.png}
      \caption{resultados de simulación para saturación del JFET.}
      \label{fig:sim.sat}
    \end{figure}
    En la figura \ref{fig:sim.sat} puede observar los resultados de la simulación. Como se puede ver, a partir de
    aproximadamente $1.5V$, el JFET llega a su corriente $I_{DSS}$. A pesar de que el voltaje $V_{DS}$ sigue aumentando, la
    corriente $I_D$ permanece prácticamente constante.

    La hoja de datos especifica una $I_{DSS}$ que va de $1mA$ a $5mA$. El modelo que se utilizo usa el mínimo de
    corriente $I_{DSS}$, lo cual se corrobora en la gráfica de la figura \ref{fig:sim.sat}, estabilizándose al rededor
    de los $2.5mA$.

  \section{Actividad de Laboratorio}
