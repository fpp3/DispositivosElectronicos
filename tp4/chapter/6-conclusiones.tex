\chapter{Conclusiones}
Este trabajo práctico se centró en el análisis del transistor de efecto de campo de juntura (JFET), un
componente fundamental dentro de la electrónica analógica y de potencia. El JFET, por su estructura y
principio de funcionamiento, permite comprender cómo el control de una unión polarizada en inversa
afecta directamente la conducción de corriente a través de un canal semiconductor, lo que lo convierte
en un dispositivo idóneo para aplicaciones de conmutación y amplificación.  

El objetivo principal del trabajo fue caracterizar experimental y teóricamente las distintas regiones 
de operación del JFET, abarcando desde el corte hasta la saturación, así como la zona óhmica 
intermedia. Para ello, se diseñaron e implementaron circuitos en un entorno de simulación (LTSpice) 
y posteriormente se replicaron en el laboratorio utilizando instrumental básico como fuentes de 
alimentación, multímetros y protoboard.  

Durante la práctica se obtuvieron curvas características $I_{DS} = f(V_{DS})$ para diferentes valores 
de $V_{GS}$, lo que permitió observar el efecto del voltaje de compuerta sobre la corriente de drenaje. 
Asimismo, se determinó la corriente de saturación $I_{DSS}$, el voltaje de estrangulamiento 
$V_{GS(off)}$ y la característica de transferencia universal, parámetros esenciales para describir 
matemáticamente el comportamiento del dispositivo.  

La comparación entre los datos experimentales, los resultados de simulación y los valores provistos 
por el fabricante en la hoja de datos permitió no solo validar los modelos teóricos, sino también 
identificar discrepancias atribuibles a tolerancias de fabricación, condiciones de medición y 
limitaciones propias de los modelos idealizados. Esta integración entre teoría, simulación y práctica 
resulta esencial para consolidar un aprendizaje completo sobre el funcionamiento real de los 
semiconductores.   
