\chapter{Conclusiones}
El análisis realizado permitió corroborar el funcionamiento del JFET en sus diferentes regiones de operación y verificar
la validez del modelo teórico. Los resultados obtenidos en simulación mostraron un comportamiento acorde a lo esperado,
mientras que las mediciones de laboratorio, si bien presentaron ligeras desviaciones, se mantuvieron dentro de los
márgenes especificados por la hoja de datos del fabricante.

Se comprobó que el modelo de Shockley describe de manera adecuada la relación entre la corriente de drenaje y la tensión
compuerta–fuente, destacando la relevancia de parámetros como $I_{DSS}$ y $V_{GS(off)}$ en la caracterización del
dispositivo. Asimismo, la comparación entre simulación, práctica y datasheet permitió evidenciar las limitaciones de los
modelos ideales frente a la variabilidad propia de los componentes reales.

En conclusión, el trabajo facilitó la comprensión del JFET como un dispositivo controlado por voltaje, reforzando la
importancia de su análisis para aplicaciones prácticas en circuitos electrónicos, especialmente en amplificación y
control de señales.
