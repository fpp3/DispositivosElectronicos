\chapter{Introducción}
  El presente trabajo práctico tiene como objetivo el estudio y análisis de los dispositivos de conmutación
  pertenecientes a la familia de los tiristores, específicamente el Rectificador Controlado de Silicio (SCR), el DIAC y
  el TRIAC. Estos componentes resultan esenciales en el control electrónico de potencia, dado que permiten regular el
  flujo de corriente mediante señales de disparo, posibilitando su aplicación en sistemas de control de iluminación,
  motores, calefactores y circuitos de regulación.

  A lo largo del desarrollo del trabajo se realizaron tanto simulaciones en LTspice como mediciones experimentales en
  laboratorio, con el fin de obtener y comparar las curvas características de cada dispositivo. Se analizaron las
  condiciones de disparo, conducción y apagado, así como los parámetros eléctricos asociados —tales como corriente y
  tensión de compuerta, corriente de mantenimiento y tensión de ruptura—.

  Asimismo, se llevó a cabo la interpretación de hojas de datos de los dispositivos empleados (TYN612M, DB3 y BT136),
  con el propósito de contrastar los valores teóricos provistos por el fabricante con los obtenidos en las experiencias.
  De esta manera, el trabajo permitió consolidar los conocimientos teóricos sobre el comportamiento de los tiristores y
  su respuesta ante diferentes condiciones de operación.
