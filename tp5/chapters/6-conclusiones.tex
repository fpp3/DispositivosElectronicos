\chapter{Conclusiones}
  El estudio realizado permitió comprender el principio de funcionamiento y las características esenciales de los
  tiristores, verificando experimentalmente los fenómenos de disparo y conducción que los definen.

  En el caso del SCR, se comprobó su activación mediante corriente de compuerta, así como la presencia de una corriente
  de mantenimiento necesaria para sostener el estado de conducción. El DIAC demostró un comportamiento bidireccional con
  un voltaje de ruptura próximo a los 30 V, mientras que el TRIAC evidenció su capacidad de conducir corriente en ambos
  sentidos, actuando funcionalmente como dos SCR conectados en antiparalelo.

  Las simulaciones y mediciones realizadas presentaron una buena correlación con las curvas teóricas, salvo ciertas
  discrepancias atribuibles a las limitaciones de los modelos SPICE utilizados. En conjunto, el trabajo permitió
  afianzar los conceptos fundamentales de los dispositivos de conmutación controlada, su modelado, y su aplicabilidad en
  circuitos de control de potencia en corriente alterna.
