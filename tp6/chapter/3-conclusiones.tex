\chapter{Conclusiones}

Durante el desarrollo del inversor CMOS surgieron desafíos que son característicos del diseño 
analógico utilizando herramientas open–source. Estos desafíos aparecieron tanto en la preparación
del entorno como en la ejecución del flujo de diseño.

Uno de los primeros desafíos fue en lograr que todas las herramientas reconocerieran adecuadamente
el PDK \textit{SkyWater SKY130}. Esto implicó resolver dependencias de compilación, ajustar rutas en
archivos de configuración como \texttt{xschemrc} y verificar manualmente la presencia de modelos 
SPICE, librerías de símbolos.

La herramienta Xschem requirió una configuración cuidadosa de variables como 
\texttt{XSCHEM\_LIBRARY\_PATH} y \texttt{search\_path} para poder acceder a los símbolos del PDK. 
Errores menores en las rutas producían la imposibilidad de cargar dispositivos esenciales del 
proceso.

En la etapa de simulación se debieron ajustar parámetros como los anchos de canal, la capacidad de 
carga, la fuente PULSE y los tiempos de análisis transitorio. Este proceso permitió obtener formas
de onda coherentes y evaluar tiempos de subida, bajada y regiones de operación de los transistores.

Herramientas como Xschem, Ngspice, Magic y Netgen poseen interfaces y flujos de trabajo diferentes
entre sí. A diferencia de otras aplicaciones comerciales más integradas, el flujo open–source 
requiere comprender cómo interactúa cada herramienta y cómo se conectan mediante archivos 
intermedios. Esto representó un gran desafio, especialmente en la depuración de errores y la 
interpretación de mensajes de simulación o verificación.

Un aspecto importante fue la necesidad de realizar múltiples iteraciones para corregir errores de
configuración, incompatibilidades entre librerías o parámetros eléctricos no apropiados. Este 
comportamiento es típico del diseño analógico y pone de manifiesto que pocas veces un circuito 
funciona correctamente en el primer intento.
Los principales desafíos se centraron en la integración del PDK, la configuración de las
herramientas y la correcta interacción entre esquemático, modelos físicos y simulaciones. Aunque 
demandó tiempo resolver estos inconvenientes, el proceso permitió comprender el flujo real de diseño 
y adquirir una visión profunda del trabajo con circuitos CMOS utilizando herramientas open–source.

Las reglas de diseño físico y las características del proceso CMOS influyeron directamente en el 
desarrollo del circuito, ya que establecen dimensiones mínimas, espaciamientos y restricciones de 
enrutamiento que deben respetarse para garantizar la fabricabilidad del diseño. Estas limitaciones 
condicionaron la elección de los tamaños de los transistores, la proporción entre NMOS y PMOS, así 
como la selección de dispositivos compatibles con el PDK. Además, las propiedades eléctricas del 
proceso (como movilidad, capacitancias y parasitismos) afectaron la respuesta dinámica del circuito
y guiaron las decisiones tomadas durante la simulación. Estas reglas aseguraron que el diseño 
esquemático, las simulaciones y el futuro layout se mantuvieran coherentes con un flujo CMOS 
realista.

Los parámetros eléctricos y físicos más críticos para cumplir con las especificaciones iniciales
fueron la ganancia de tensión, el punto de operación de los transistores, la corriente de 
polarización y el margen de salida, ya que determinaron directamente el funcionamiento lineal del 
circuito. También resultaron fundamentales la relación entre los anchos y largos de canal (W/L), las
capacitancias parásitas asociadas al proceso CMOS y las resistencias internas, porque afectaron la 
velocidad de respuesta, el slew rate y la estabilidad del circuito. Finalmente, el correcto ajuste 
del umbral de los MOSFET y el control del consumo de potencia fueron esenciales para garantizar que
el desempeño simulado coincidiera con los requisitos planteados.

La validación del funcionamiento del circuito antes del diseño físico se logró mediante una serie de 
simulaciones eléctricas realizadas en distintas etapas. Primero, se efectuaron simulaciones de 
\textit{DC} para verificar los puntos de operación y asegurar que los transistores trabajaran en la 
región apropiada. Luego, se realizaron simulaciones de \textit{transitorio} para analizar la 
respuesta temporal y confirmar que el circuito cumpliera con los tiempos de establecimiento y el 
comportamiento dinámico esperado. Finalmente, se llevaron a cabo simulaciones de \textit{AC} y 
barridos paramétricos para evaluar la ganancia, el ancho de banda y la estabilidad frente a 
variaciones. Este conjunto de análisis permitió identificar problemas tempranos, ajustar dimensiones
de dispositivos y confirmar que el diseño lógico y eléctrico era funcional antes de avanzar al 
diseño físico.
